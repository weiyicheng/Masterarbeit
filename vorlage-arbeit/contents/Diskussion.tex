% ===========================================================
\chapter{Diskussion} \label{diskussion}
% ===========================================================
Es gibt schon viele Forschungen über die \acl{OCRs}, und der Begriff, die Motivationen und die Auswirkungen werden durch die Studien breit reichlich studiert. Aber es gibt nicht viele Ergebnisse über die kulturellen Einflüsse. Gibt es Unterschied in den \ac{OCRs} aufgrund der kulturellen Einflüsse? Wenn ja, welche Unterschiede, und warum? Diese Fragen sind noch zu beantworten. Besonders bei der qualitativen \ac{OCRs}, also der schriftliche Teil, gibt es noch wenige Studien darüber.In dieser Arbeit werden die \ac{OCRs} nach \citet{Shrihari2012} in den quantitativen und qualitativen Teil getrennt und jeder Teil wird drei Attribute gegeben: Volumen, Valenz, und Varianz. Außerdem haben die beiden Teile eine gemeisame Attibute: Korrelationskoeffizient. Durch diesen digitalen Attribute kann man die statistischen Maßnahmen benutzen, um die allgemeine Unterschiede hinter der Rohdaten herauszufinden.

Für diese Attribute werden vier Hypothesen gebildet, um die Unterschiede zwischen China und Deutschland zu entdecken. Hypothese \ref{hyp:1}, \ref{hyp:2} und \ref{hyp:3} sind über das Volumen, die Valenz, und die Varianz der \ac{OCRs} sowohl in dem quantitativen Aspekt als auch dem qualitativen Aspekt. Hypothese \ref{hyp:4} sind über den Zusammenhang der quantitativen und qualitativen \ac{OCRs}. Diese vier Hypothese versuchen, die statistische Unterschiede herauszufinden, durch die Attribute. Tabelle \ref{tab:ergebnis_hypothese} zeigt die zusammenfassenden Ergebnisse.

\begin{table}[h]
\centering
\resizebox{\textwidth}{!}{%
\begin{tabular}{|c|c|c|c|}
\hline
Hypothese & Ergebnisse              & Quantitativer Aspekt           & Qualitativer Aspekt         \\ \hline
\ref{hyp:1}         & Volumen                 & China \textgreater Deutschland & China \textless Deutschland \\ \hline
\ref{hyp:2}          & Valenz                  & China \textgreater Deutschland & China \textless Deutschland \\ \hline
\ref{hyp:3}          & Varianz                 & China \textless Deutschland*   & China \textless Deutschland \\ \hline
\ref{hyp:4}          & Korrelationskoeffizient & \multicolumn{2}{c|}{China \textless Deutschland*}            \\ \hline
\end{tabular}
}
\caption[Die zusammenfassenden Ergebnisse von Hypothesen]{Die zusammenfassenden Ergebnisse von Hypothesen (*: Ausnahme ist Puma) (Quelle: Eigene Darstellung)}
\label{tab:ergebnis_hypothese}
\end{table}


In den folgenden Teilen ist es zu versuchen, die kulturellen Gründen für diese Ergebnisse möglicherweise nach den theoretischen Grundlagen herauszufinden. Und auch, wird es diskutiert, welche Auswirkungen es auf den Kunden \ac{bzw.} den Unternehmen gibt.
%%===========================================================
\section{Qualitativ gegenüber Quantitativ}
%%===========================================================
Laut \citet{Lam2009} sind die Menschen zurückhaltend bei dem Informationsaustausch, wenn sie in der Kultur mit großer Machtdistanz leben. Und nach \citet{Lam2009, liu2001relationships, dawar1996cross, money1998explorations} können die Menschen mit einer hohe Unsicherheitsvermeidung ein höheren Niveau der Meinungsaustausch sowie Meinungsssucht haben werden. In Abbildung \ref{fig:sechsdimensionen} sieht man schon, dass es in China sehr große Machtdistanz gibt, und die deutsche Unsicherheitsvermeidung ist relativ höher als in China. Deshalb wird es in Hypothese \ref{hyp:1} erwartet, dass das chinesische Volumen kleiner als das von Deutschen ist, sowohl bei den quantitativen als auch bei den qualitativen \ac{OCRs}. Diese Hypothese meint, dass die deutschen Kunden langer und mehrerer schreiben würden während die chinesischen Kunden kurzer und weniger schreiben würden.

Aber die Tatsache zeigt anders aus. Es stimmt, dass die Deutschen langer schreiben und die Chinesen kurzer schreiben. Aber die Chinesen schreiben viel lieber als die Deutschen. Die Menge der chinesischen \ac{OCRs} ist 15fache wie die der deutschen \ac{OCRs} bei der Untersuchungsobjekten in dieser Arbeit. In dieser Situation wird es zusammengefasst, dass die Deutschen qualitativ schreiben wollen aber die chinesischen \ac{OCRs} sind quantitativ.

Es ist klar, das wer zurückhaltend bei dem Informationsaustausch ist, kurzer schreiben wird. Diese Zurückbehaltung ist wegen der großen Machtdistanz in Kultur nach \citet{Lam2009}. Die Deutschen, die mit kleiner Machtdistanz in Kultur leben, sind offener als die Chinesen und vermeiden die Unsicherheit lieber \citep{Lam2009, liu2001relationships, dawar1996cross, money1998explorations}, damit wollen sie mehr Information austauschen, deswegen schreiben sie langer.

Aber die Frage, warum das chinesische quantitative Volumen so groß ist, steht noch in dem theoretischen Bereich offen. Es sollte mindestens eine extra Motivation von der Chinesen für das Schreiben des Reviews im Vergleich zu der Deutschen geben. Es wird vorgeschlagen, dass diese Motivation aus der kollektivistischen Kultur ist. Bei der kollektivistischen Kultur ist, ``alle machen, dann mache ich auch'' normal. Es entsteht die Motivation, ein Review für das Produkt zu schreiben, wenn das Produkt schon viele Reviews hat. Aber die Motivation ist nicht so stark genug für ein langes Review. Die Chinesen möchten das zeigen, dass sie noch in der Gruppe sind, in deren alle Reviews schreiben. Diese in-Gruppenmitgliedschaften sind für sie wichtig. Aber die Deutschen, als die Mitglieder einer individualistischen Kultur, sind unabhängig von anderen, und betrachten die Getrenntheit und Entfernung von in-Gruppen. \citep{singelis1994measurement}

Für die anderen chinesischen Konsumenten, sind die chinesischen \ac{OCRs} schlechter als die Deutschen, weil die Kunden, die \ac{OCRs} geschrieben haben, wenige Information gegeben haben, obwohl die Menge groß ist. Wenn man nur einfach ``gut'' oder ``Okay'' geschrieben hat, wissen die anderen Leute nicht, worüber das Review ist. Deshalb müssen sie noch mehr lesen, um die richtigen und wichtigen Informationen zu finden. Die kurzen \ac{OCRs} werden als ``Noise Text'' von den anderen Kunden oder potentialen Kunden genannt. In dieser Weise sollte der Kunde auch wissen, dass es keine Hilfe ist, wenn er oder sie nur ein kurzes Review geschrieben hat. Der Kunde sollte vermeiden, das kurzes Review zu schreiben, sondern möglicherweise die Attribute des Produkts zu beschreiben, damit er oder sie den anderen helfen könnte. 

Andererseits gibt es diese Situation für die deutschen \ac{OCRs} kaum. Aber in den deutschen \ac{OCRs} gibt es noch Probleme. Zu langes Review ist auch sinnlos, weil die anderen Menschen keine lange Zeit haben, um das lange Review durchzulesen. Deshalb könnte das lange Review nicht so viel helfen. Das Review sollte nicht zu kurz oder zu lang sein, aber die Attribute des Produkts sollte als Stichpunkte beschrieben werden. Zum Beispiel, ist das ein gutes Review, wenn man über die Vorteile und Nachteile der Kleidung in den Attributen: ``Ästhetik'', ``Performance'' und ``äußere Attribute'' geschrieben hat.

Es war auch eine Herausforderung, die die Unternehmen oder die Betreiber der Plattform sie meistern müssen, die ``Noise Text'' zu vermeiden oder wesentlich sich nicht an der ersten Stelle stehen lassen. Beim Vermeiden kann man einfach die minimale Anzahl der Wörter beschränken, aber wie gesagt, die Motivation ist nicht so stark, damit kann man einfach das Schreiben des Reviews aufgeben. Deshalb machen die Unternehmen gerne die \ac{OCRs} nochmal ordnen, die nicht in zeitlicher Reihenfolge, sondern in nützlicher Reihenfolge sind. 

Die chinesische Plattform ``Tmall'' ordnet die \ac{OCRs} durch einen bestimmten, damit kann man die längere \ac{OCRs} mit Photos erst lesen. Die deutsche Plattform ``Amazon'' ordnet die \ac{OCRs} durch ``Hilfreich''. Wenn die anderen Leute denken, dass dieser Kommentar für sie hilfreich ist, drucken sie ``Ja'' und die anderen sehen schon, dass dieser Kommentar einem Mensch hilft. Und der Kommentar kann nach vorne stehen, damit mehreren Menschen den lesen könnten.
%%===========================================================
\section{Positiv gegenüber Neutral} \label{sec:diskussion:2}
%%===========================================================
Nach \citet{liu2001relationships} geben die Personen aus geringerer Unsicherheitsvermeidung häufiger negative Kommentaren, als die Personen in Kultur mit höherer Unsicherheitsvermeidung, wenn sie schlecht bedient werden. Und laut \citet{donthu1998cultural}, geben die Menschen in Kulturen mit hoher Unsicherheitsvermeidung im Allgemeinen positivere Kommentaren, als die mit niedriger Unsicherheitsvermeidung. Und Wie aus Abbildung \ref{fig:sechsdimensionen} ersichtlich, sind die Unsicherheitsvermeidung der Deutschen relativ höher als die von Chinesen. Deshalb wird es in Hypothese \ref{hyp:2} vorgeschlagen, dass die deutschen Valenzen größer als die chinesischen Valenzen der \ac{OCRs} in dem quantitativen und auch in dem qualitativen Aspekt sein sollten. Also, die deutschen \ac{OCRs} sollten positiver als die chinesischen \ac{OCRs} sein.

Aber es stimmt nur teilweise in der Tat. Die deutschen qualitativen \ac{OCRs} sind tatsächlich positiver als die chinesischen durch die Sentiment Analyse, aber in dem quantitativen Aspekt präsentieren die chinesischen \ac{OCRs} positiver. Im Durchschnitt geben die Deutschen für die Untersuchungsobjekte nur 4,51 Sterne im Vergleich zu 4,88 Sterne von den Chinesen (Maximal 5 Sterne). Die durchschnittliche Valenz von den deutschen qualitativen \ac{OCRs} ist 0,6702, schon relativ hoch (beim einzelnen Wort bis 1, und in dem Lexikon von \citet{Remus2010} ist 0,6702 zwischen ``zuvorkommend, 0,6669'' und ``romantisch, 0,6965''). Die chinesische durchschnittliche Valenz ist 0,2525 und es ist im Lexikon von \citet{Remus2010} zwischen ``fröhlich, 0,2501'' und ``gefallen, 0,2578''. Daher sind die deutschen \ac{OCRs} positiv in dem qualitativen Aspekt und die chinesischen qualitativen \ac{OCRs} sind neutral.

Die kulturelle Begründung der positiveren deutschen qualitativen \ac{OCRs} ist nach den Forschungen von \citet{donthu1998cultural} klar, und laut \citet{liu2001relationships} sollen die chinesischen \ac{OCRs} negativer als die deutschen sein, aber dieses Phänomen, dass die chinesischen Kunden viel lieber 5 Sterne geben (siehe in Abbildung \ref{fig:struktur_allgemein}), wird noch nicht begründet. Möglicherweise ist das, dass es eine wirtschaftliche Anreize für die chinesischen Kunden gab, die 5 Sterne gegeben haben. Wenn man in den kulturellen Dimensionen denkt, wird es vermutet, dass die kollektivistische Kultur der Chinesen das Phänomen erzeugt. Nach \citet[p. ~65]{hofstede1998masculinity}, werden die individuelle Entscheidungen (zum Beispiel: wie viele Sterne gebe ich) in kollektivistischer Kultur in Konsens mit der Gruppe gemacht und es gibt keine rein individuelle Entscheidungen zu treffen. Aber im Gegensatz dazu sind die Menschen in individualistischen Kulturen eher selbst als autonome zu denken und legen die individuellen Interessen an erster Stelle \citep{shweder1990defense}. 

Das heißt, dass die chinesische einzelne Valenz von dem quantitativen Review (also Sterne) von den bereits existierten \ac{OCRs} groß beeinflusst wird, während die einzelne deutsche Valenz individuell ist. Wenn es schon viele ``5 Sterne'' \ac{OCRs} gab, könnte es sein, dass die chinesischen Kunden ``5 Sterne'' geben wollen, obwohl sie nicht zufrieden sind. Nach der Hinsicht in die Rohdaten, wird das Beispiel gefunden: ``
\begin{CJK*}{UTF8}{gbsn}
	老公说质量不怎么样。
\end{CJK*}
'' Die Kundin meint, dass ihrer Mann sagt, dass die Qualität des T-Shirts nicht so gut ist, trotzdem sie ``5 Sterne'' gegeben hat. Es ist nur ein Beispiel, aber es gibt noch einige ähnlichen Situationen in den chinesischen Daten.

Die chinesischen Kunden sollten vorsichtiger sein, als die Deutschen, weil die anderen chinesischen Kunden ``5 Sterne'' gegeben haben könnten, trotzdem das Produkt nicht gut sein könnte. Liest man nur die Sterne, ist es nicht genug, sondern nochmal das Text durchzulesen, besonders bei der chinesischen Kunden.

Für die Unternehmen, ist es wichtig zu wissen, dass es Probleme geben könnte, obwohl die quantitativen Valenzen gut scheinen. Sie sollten auch das Text durchlesen, und schnell reagieren, um die Probleme zu lösen. 

\citet{Luo2014} haben herausgefunden, dass durch die stärkere Wirkung der zweiseitigen Informationen (im Vergleich zu einseitigen Informationen) die Wahrnehmung von Glaubwürdigkeit der Informationen der \ac{eWOM} Leser gestärkt wird, wenn die Leser mit individualistischen Kulturen sind, verglichen mit denen, die mit kollektivistisch kulturellen Orientierung sind. Deshalb sollte die Reihenfolge der \ac{OCRs} in Deutschland und China unterschiedlich sein. Die Betreiber von Amazon sollten nicht nur die positiven Bewertungen sondern auch einige negativen Bewertungen an den ersten Positionen stellen, damit für die deutschen Kunden ist die Glaubwürdigkeit der \ac{OCRs} höher. Aber die Betreiber von Tmall brauchen vielleicht nicht so zu machen.
%%===========================================================
\section{Dezentralisiert gegenüber Zentralisiert}
%%===========================================================
Laut \citet{hofstede2013interkulturelle} ist die Ungleichkeit bei großer Machtdistanz erwünscht und wird erwartet, deswegen sollten die \ac{OCRs} dezentralisiert sein. Aber nach der Forschung der \citeauthor{Luo2014} sollten die \ac{OCRs} konsistenter und dichter beim Kollektivismus sein. Die Chinesen leben in der kollektivistischen Kultur mit großer Machtdistanz. Die Deutschen leben in individueller Kultur mit geringer Machtdistanz. Weil die vorherigen Forscher die Dimension ``Individualismus gegenüber Kollektivismus'' als eine Tiefenstruktur denken \citep{grennfield2000approaches,sia2009web, triandis2001individualism}, wird es in Hypothese \ref{hyp:3} vorgeschlagen, dass die chinesischen \ac{OCRs} zentralisiert und deutschen \ac{OCRs} dezentralisiert sein würden.

Die Tatsache stützt die Hypothese. In dem quantitativen und auch qualitativen Aspekt, sind die Varianzen von den chinesischen \ac{OCRs} kleiner als die von den deutschen im Allgemeinen. Für das jeweilige Produkt, ist die Situation fast gleich, aber es gibt eine Ausnahme bei den quantitativen \ac{OCRs} von Puma. Der Grund von dieser Außerordentlichkeit ist noch nicht klar. Vielleicht ist es wegen der Machtdistanz nach \citet{hofstede2013interkulturelle}, aber dafür ist nicht der Schatten eines Beweises zu erbringen. Durch diese Ergebnisse, ist es auch ersichtlich, dass die Dimension ``Individualismus gegenüber Kollektivismus'' eine Tiefenstruktur und wichtigste Dimension ist, wie die Forscher schon vorher gemacht haben. 

Die chinesischen Kunden sollten vor dem Lesen der \ac{OCRs} wissen, dass sie viel dichter sein würden. Damit können die Kunden nicht einfach von den dicht guten \ac{OCRs} auf den Kaufentscheidungen beeinflussen. Man sollte mehr lesen, um die Schwäche und Stärke des Produkts oder andere Benutzererfahrungen zu suchen, damit sind die Informationen eher umfassend. Die deutschen Kunden sollten auch das wissen, dass die \ac{OCRs} individuell und dezentralisiert sind. Das heißt, dass die \ac{OCRs} nicht nur positiv sein könnten. Individuelle Erfahrung oder Wahrnehmung könnte nicht selbst anpassend sein, wie zum Beispiel schreibt ein Kunde einfach ``Die Farbe mag ich nicht'' und deswegen hat er 1 Stern gegeben oder so weiter. 

Die Unternehmen sollten auch diese Verteilung erkennen. In China sind die \ac{OCRs} zentralisiert und dicht. Die Unternehmen versuchen, das Zentrum der \ac{OCRs} möglicherweise in positive Seite zu entwickeln. Die wichtige Schwäche werden durch die \ac{OCRs} häufig gezeigt und diese Probleme sollten die Unternehmen erst lösen. In Deutschland sind die \ac{OCRs} dezentralisiert, deswegen müssen die Unternehmen selbst zusammenfassen und die Stärke und Schwäche wissen, um die besseres Produkt zu bieten.

Wenn die Betreiber der Plattformen die Technologie anbieten könnten, die Kunden fokussierenden Aspekte oder die Attituden der Kunden aus den \ac{OCRs} automatisch zusammenzufassen, wäre es besser. Damit die Kunden und Unternehmen nicht viel lesen und einfach erkennen das, worüber die Stärke und Schwäche des Produkts sind. Tmall von China hat diese Zusammenfassung schon gemacht, aber bei Amazon gibt es noch nicht.
%%===========================================================
\section{Abhängig gegenüber Unabhängig}
%%===========================================================
Obwohl es schon viele Forschungen über die \ac{OCRs} gibt, bleibt der Zusammenhang zwischen den quantitativen und qualitativen \ac{OCRs} heutzutage nicht in der Sicht der Wissenschaftler. Gibt es bestimmten Zusammenhang? Wenn ja, ist der Zusammenhang positiv oder negativ? Gibt es kulturelle Unterschiede von dem Zusammenhang? Diese Fragen werden noch nicht beantwortet. Hypothese \ref{hyp:4} schlagt vor, dass es den Zusammenhang gibt, und der deutsche Zusammenhang zwischen quantitativen und qualitativen \ac{OCRs} ist stärker als den von chinesischen \ac{OCRs}. Dieser Zusammenhang wird durch den Korrelationskoeffizient und Rangkorrelationskoeffizient gemessen.

Durch die Teste wird die Hypothese \ref{hyp:4} in den meisten Fällen gestützt. Es gibt positiven Zusammenhang zwischen den quantitativen und qualitativen \ac{OCRs} und der Zusammenhang von den deutschen \ac{OCRs} ist stärker als der von den chinesischen \ac{OCRs}. Diese Ergebnisse werden von den Ergebnissen der Hypothese \ref{hyp:2} gestützt, weil die chinesische quantitative Valenz größer als die deutschen, aber in dem qualitativen Aspekt ist die deutsche Valenz größer. Wie es in Abschnitt \ref{sec:diskussion:2} durch das Beispiel schon gezeigt wird, sind die chinesischen quantitativen \ac{OCRs} manchmal unabhängig von den qualitativen \ac{OCRs}, aber die Deutschen geben die Sterne mit der Abhängigkeit davon, was sie schreiben wollen.

Es gibt in den Ergebnissen auch eine Ausnahme von Puma. Der deutsche Zusammenhang ist nicht mehr positiv sondern negativ. Gemäß von dem Korrelationskoeffizient  ist dieser Zusammenhang fast null. Der Grund der Außerordentlichkeit ist noch nicht bekannt.

Die Frage, warum die Chinesen ohne oder mit kleinerer Abhängigkeit Sterne geben und \ac{OCRs} schreiben, ist noch offen. Vielleicht gibt es wirtschaftliche Anreize für die Kunden, die ``5 Sterne'' gegeben haben. In dem kulturellen Aspekt, werden die Sterne von chinesischen Kunden von den bereits existierten \ac{OCRs} größer beeinflusst, als die Deutschen, wegen des Kollektivismus. 

Die Kunden schreiben die qualitativen \ac{OCRs} (Text), um die wirkliche Situation zu beschreiben und das eigene Gefühl auszudrucken. Je mehr man Text schreibt, desto mehrere Informationen werden über die Attribute des Produkts beschrieben. Zum Beispiel: Die allgemeine Attributkategorien für Kleidung sind Ästhetik, Performance, und äußere Attribute. Die qualitativen \ac{OCRs} werden meistens über diese drei Attributkategorien geschrieben. Wie sich bereits aus Abbildung \ref{fig:top10} ersehen lässt, schreiben die deutschen Kunden mehr über die Performance (``gross'', ``angenehm'', ``passt'', ``trage'' in Top 10 Wörtern) und die chinesischen Kunden schreiben mehr über die Performance (``tragbaren'' und ``gross'') und die äußere Attribute (``schnell'', ``echt'' und ``logistics'' in Top 10 Wörtern). Diese Informationen sind individuell und das könnte nicht einfach von anderen Leute beeinflusst werden. In diesem Sinne sind die qualitativen \ac{OCRs} glaubwürdiger als die quantitativen \ac{OCRs}. 

Diese Ergebnisse sind wichtig für die Kunden und die Unternehmen. Die chinesischen Kunden sollten erkennen, dass die Sterne nicht so glaubwürdig sind, weil sie einfach beeinflusst werden könnten. Mehr zu lesen und die Attribute über die Kleidung zusammenzufassen ist es besser als Glauben an den Sternen, weil der Zusammenhang nicht stark ist. Für die Deutschen sind das Problem nicht so groß wegen des stärkeren Zusammenhang, aber für die Sicherheit ist es auch besser, die qualitativen Teile der \ac{OCRs} durchzulesen.

Die Unternehmen sollten auch nicht durch die Mehrheit der ``5 Sterne'' verwirrt sein. Sie sollten die qualitativen \ac{OCRs} durchlesen und die Probleme schnell reagieren, damit sie besseres Produkt und Service anbieten könnten. Und wie vorher schon gesagt, wird es besser sein, dass die Betreiber der Plattform die Webseite so organisieren, die Kunden fokussierenden Attribute automatisch zusammenzufassen.
