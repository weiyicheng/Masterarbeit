% Copyright (C)  2013 Jana Traue (jana.traue[at]tu-cottbus.de)
%
% Permission is granted to copy, distribute and/or modify this document
% under the terms of the GNU Free Documentation License, Version 1.3
% or any later version published by the Free Software Foundation;
% with no Invariant Sections, no Front-Cover Texts, and no Back-Cover Texts.
% A copy of the license is included in the file entitled "LICENSE".
%
% =============================================================================
% LATEX chapter
% =============================================================================
%
% =============================================================================
% =============================================================================
\chapter{Demo-Kapitel (Deutsch)}
% =============================================================================
%
Herzlich Willkommen in der wunderbaren Welt von \LaTeX{}.
Dieses Dokument soll als Grundlage für komplexere Dokumente, wie Bachelor-
oder Masterarbeiten, dienen.\citep{Li2011}
Komplex bezieht sich dabei nicht unbedingt auf den Inhalt - für den bleibt
der Author selbst verantwortlich - sondern auf die Dokumentstruktur.
Um sinnvoll mit dieser Vorlage umgehen zu können, wird empfohlen, die
nachfolgenden Abschnitte mindestens einmal durchzulesen.
Ganz am Ende wird dann auch verraten, wie man diesen Text aus seiner
Arbeit wieder entfernen kann.
%
% =============================================================================
\section{Generelles zu \LaTeX{}}
% =============================================================================
%
% -----------------------------------------------------------------------------
\subsection{Übersetzen}
% -----------------------------------------------------------------------------
%
Im obersten Ordner befindet sich eine Datei \code{main.tex}.
Dieses wird übersetzt mit:
%
\begin{lstlisting}[language=bash]
pdflatex main.tex
\end{lstlisting}
%
Oftmals ist es notwendig, diesen Befehl zweimal auszuführen, damit
Verweise richtig eingebaut werden.
%
\par
%
Das Literaturverzeichnis muss extra übersetzt werden, mit Hilfe von:
%
\begin{lstlisting}[language=bash]
bibtex main
\end{lstlisting}
%
und danach einem erneutem \code{pdflatex}.
%
\par
%
Damit eure Betreuer nicht von allen Studenten Dateien mit dem Namen
\code{main.pdf} bekommen, solltet ihr die \code{main} als erstes
eindeutig kennzeichnen, beispielsweise mit eurem Namen.
%
\par
%
Damit ihr die oben genannten Befehle nicht jedesmal tippen m\"usst, gibt
es ein Makefile.
Mit Hilfe von \code{make} wird dann sowohl \code{pdflatex} als auch
\code{bibtex} angestossen.
Das Aufrufen von \code{make clean} sorgt daf\"ur, dass tempor\"are Dateien
wieder gel\"oscht werden und ist vor dem Einchecken in eine Versionsverwaltung
sinnvoll.
Bevor ihr es zum Aufräumen verwendet, werft einen Blick in das Makefile.
Es l\"oscht n\"amlich alle Dateien mit vorgegebenen Endungen.
Im schlimmsten Fall haben eure Dateien eine Endung, die dort angegeben ist,
dann verschwinden sie im Nirwana.
Habt ihr eurer Dokument von \code{main} umbenannt, so m\"usst ihr das
Makefile anpassen.
%
% -----------------------------------------------------------------------------
\subsection{Ordnerstruktur der Vorlage}
% -----------------------------------------------------------------------------
%
Es ist zwar möglich, euren gesamten Text in eine einzige Datei zu schreiben, 
aber wer sieht da dann noch durch?
Deshalb ist in dieser Vorlage eine komplexere Struktur vorgegeben, die sich
in den letzten Jahren bewährt hat.
Neben der \code{main} befinden sich im obersten Ordner folgende weitere
Ordner:
\begin{itemize}
	\item \code{bib}: Literaturverzeichnis
	\item \code{contents}: tex-Dateien, also euer eigentlicher Text
	\item \code{contents/global}: tex-Dateien, die ihr vllt. einmal anschauen und
        anpassen müsst (das Deckblatt z.B.), die aber beim Schreiben des
        eigentlichen Inhalts nicht verändert werden
	\item \code{figures}: Bilder 
\end{itemize}
%
%
% .............................................................................
\subsubsection{Fehlende \LaTeX{} Pakete}
% .............................................................................
%
Wie ihr auf eurem System \LaTeX{} und die notwendigen Pakete
installiert, müsst ihr selbst herausfinden.
%
Wenn ihr eine Fehlermeldung ala:
\begin{lstlisting}[language=bash]
! LaTeX Error: File '<package name>.sty' not found.
\end{lstlisting}
bekommt, fehlt euch ein Paket.
%
Manchmal werden Pakete benötigt, die nicht in den Standard-Distributionen dabei
sind\ldots oder euch fehlen die Rechte zum Installieren, beispielsweise bei
unseren Pool-Rechnern.
Eine Lösung besteht darin, die Pakete von
\href{http://www.ctan.org/}{CTAN} herunterzuladen und entwedet nach 
\code{/usr/share/texmf/tex/latex/} zu kopieren (was im Pool dann nicht klappt)
oder sie in den Ordner \code{/home/yourName/texmf/tex/} zu legen.
%
Danach muss noch ein
\begin{lstlisting}[language=bash]
texhash .
\end{lstlisting}
in eurem Home-Verzeichnis ausgeführt werden.
%
% -----------------------------------------------------------------------------
\subsection{Bewährte Code-Konventionen}
% -----------------------------------------------------------------------------
%
\begin{itemize}
    \item Legt für jedes Kapitel euerer Arbeit eine extra Datei oder einen
        extra Ordner an.
        Das hilft dabei die Übersicht zu behalten und zwingt auch zu einer
        klaren Strukturierung.
    \item Markiert Abschnitte und Unterabschnitte um sie leicht wiederzufinden.
        In dem Quellcode dieses Dokuments sind beispielsweise Abschnitte
        mit gestrichelten Linien markiert.
    \item Kennzeichnet Code in eurem Text, wie etwa \code{main.tex} oben
        mit speziellen \LaTeX-Befehlen, wie \code{$\backslash$code\{\}}.
        Das ist kein Standard-Befehl, sondern nur ein neues Kommando,
        was den Text im Teletype-Stil erscheinen laesst, aber das Aussehen
        lässt sich leicht ändern.
        Für komplexere Code-Abschnitte ist Abschnitt \ref{demo:examples:listings}
        da.
    \item Verwendet aussagekräftige Labels für Verweise. 
        Bringt im Namen des Labels z.B. die Art des Elements unter, auf das
        ihr verweist, wie:
        \code{$\backslash$label\{chapter:introduction\}} oder
		\code{$\backslash$label\{figure:introduction:section:name\}}. 
        Das hilft beim Vermeiden von Konflikten.
    \item Begrenzt eure Zeilenlänge auf 80 oder 90 Zeichen, so dass ihr nach
        einem Wechsel des Editors nicht ewig mit dem Setzen von Umbrüchen
        beschäftigt seid.
\end{itemize}
%
% -----------------------------------------------------------------------------
\subsection{Notwendige Anpassungen der Vorlage}
% -----------------------------------------------------------------------------
%
Die Vorlage kann sowohl für Arbeiten in Deutsch als auch in Englisch verwendet
werden.
Wer auf Deutsch schreibt, sollte die als Englisch markierten Stellen im Quellcode
ignorieren.
Die folgenden Anpassungen der Vorlage sind zwingend notwendig:
%
\paragraph{titlepage.tex}
%
Ändert den Titel des Dokuments in dem pdf bookmark Element.
Fügt Lehrstuhl und den Namen des Professors ein.
Der Typ eures Dokuments ist beispielsweise ``Bachelorarbeit''.
Füllt Titel, Name, Matrikelnummer und Studiengang aus.
Die Namen der Betreuer sollten auch eingetragen werden.
%
\paragraph{header.tex}
%

Eine Beschreibung der verwendeten 
\href{http://www.komascript.de/}{KOMA-Script}-Optionen findet ihr am
Dokumentanfang. 
Danach stellt ihr ein, ob ihr Deutsch oder Englisch als Sprache verwendet.
Wie schon erwähnt, sind einige der Dateien schon für beide Sprachen 
vorbereitet.
Passt auf, dass bei den Sprachblöcken keine Zeichen (auch keine Leerzeichen)
am Anfang der Zeile erlaubt sind:
\begin{lstlisting}[language={[LaTeX]TeX}]
\begin{de}
\end{de}
\end{lstlisting}
Verstösst man gegen diese Regel, gibt es merkwürdige Fehlermeldungen:
\begin{lstlisting}[language=bash]
Runaway argument?
! File ended while scanning use of \next.
<inserted text> 
                \par
\end{lstlisting}
Vergesst nicht, \code{cleanup.sh} nach dem Sprachwechsel auszuführen.

Füllt die Informationen aus, die vom hyperref Paket benötigt werden.
Die Hyphenations am Ende sagen {\LaTeX}, wie bestimmte Worte getrennt werden sollen.
%
\paragraph{acronyms.tex}
%
Hier werden Abkürzungen und ihre Langformen angegeben.
Verwendet man im Text eine Abkürzung, wie CPU, so schreibt man beim ersten
Auftreten die Langform aus und die Abkürzung dahinter: 
die Central Processing Unit (CPU).
Schreibt man dies manuell aus und strukturiert hinterher seinen Text um, 
so ist nicht ausgeschlossen, dass man die Langform verschieben muss.
Es ist deshalb eine gute Idee, das ganze von \LaTeX{} verwalten zu lassen.
Im Abkürzungsverzeichnis werden die Kurz- und Langformen angegeben und
im Text entstehen dann ein \ac{BTU} und \ac{BTU}, beide durch den selben
Befehl: \code{$\backslash$ac\{BTU\}}.
Im Verzeichnis muss angegebenen werden, welches das längste Akronym ist, damit
die entstehende Tabelle richtig ausgerichtet ist.
Ausserdem steht in der Datei, wie man z.B. den Plural von Abkürzungen erzeugen kann.
%
% -----------------------------------------------------------------------------
\subsection{Tool-Unterstützung}
% -----------------------------------------------------------------------------
%
\paragraph{Texmaker}
%
\href{http://www.xm1math.net/texmaker/}{Texmaker}
gibt es für verschiedene Plattformen.
Es unterstützt das Kompilieren und Anzeigen des Dokuments mit Hilfe von
Shortcuts.
Um es zu benutzen, muss einige Pfade in den Einstellungen angepasst werden.
Ist die \code{main.tex} geöffnet, sollte sie danach als Master-Datei in den 
Optionen eingestellt werden.
Leider muss das nach jedem Start erneut eingestellt werden.
Ist ein Kapitel geöffnet, so kann das Dokument mit \code{F6} übersetzt werden.
Bibtex wird mit \code{F11} gestartet und das PDF wird mit \code{F7} angezeigt.
Für die Rechtschreibprüfung muss ein Wörterbuch in den Einstellungen angegeben
werden.
%
\paragraph{gVim}
wurde zur Erstellung dieser Vorlage verwendet.
Man kann externe Befehle via
\begin{lstlisting}[language=bash]
:!<command>
\end{lstlisting}
starten, also z.B.:
\begin{lstlisting}[language=bash]
:!pdflatex main.tex
\end{lstlisting}
und muss damit nicht zu einem Terminal wechseln.
Wenn ihr nicht wisst, wie man in den Kommandomodus wechselt, solltet ihr 
auf den Einsatz von gVim lieber verzichten.
%
\par
%
Auch mit gVim kann man eine Rechtschreibprüfung durchführen.
Sie kann mit
\begin{lstlisting}[language=bash]
:set spell
\end{lstlisting}
angestellt werden.
Danach wird mit
\begin{lstlisting}[language=bash]
:set spelllang=<lang>
\end{lstlisting}
die Sprache ausgewählt, z.B.
\begin{lstlisting}[language=bash]
:set spelllang=de
\end{lstlisting}
Falsch geschriebene/unbekannte Worte werden dann im Quellcode hervorgehoben.
Ist der Cursor auf so einem Wort, kann mit \code{z=} eine Liste von
Vorschlägen geöffnet werden.
Wenn das Wort nicht vorkommt, aber dennoch richtig geschrieben ist, so 
kann es mit \code{zg} in das Wörterbuch aufgenommen werden.
In der Hilfe-Datei gibt es mehr Informationen:
\begin{lstlisting}[language=bash]
:help spell
\end{lstlisting}
%
%
% \paragraph{Notes}
% %
% Handwritten notes are awful. They can not be included in revision control 
% systems.
% Paper might get lost. You might not be able to decipher the text.
% Therefore, convince your consultants to provide digital notes.
% Several free tools for annotating pdf documents exist. For Windows there is 
% \href{http://www.foxitsoftware.com/pdf/reader/}{Foxit Reader}. 
% The build-in Preview.app supports notes on Mac OS, but they are hard to read 
% using any other tool. 
% An alternative is \href{http://skim-app.sourceforge.net/}{Skim}.
% Still looking forward to find a Linux application besides 
% \href{http://okular.kde.org/}{Okular}. 
%
% -----------------------------------------------------------------------------
\subsection{Weitere Hinweise}
% -----------------------------------------------------------------------------
%
%
\paragraph{Inhalt vs Layout}
%
Das wichtigste bei eurer Arbeit ist der Inhalt.
Während ihr den schreibt, solltet ihr nicht versuchen, ein perfektes
Layout zu erreichen.
Das kann sehr aufwendig werden und auch schnell wieder hinfällig sein, 
wenn neuer Text hinzukommt.
Kümmert euch deshalb erst ganz am Ende um das Layout.
%
\paragraph{Check Logs}
%
{\LaTeX} erzeugt ein Log-File bei jedem Kompilieren.
Prüft es am Ende und behebt alle Warnings.
%
\paragraph{Verwendet eine Rechtschreibprüfung!}
Es gibt nichts Ärgerlicheres, als wenn euer Betreuer nicht auf den Inhalt 
der Arbeit achten kann, weil er eure Rechtschreibung korrigieren muss.
Davon haben beide Seiten nichts.
%
\par
%
Wie versprochen wird nun noch verraten, wie man dieses Kapitel aus seinem Dokument
entfernt.
Dazu muss nur in der \code{main.tex} die Zeile mit dem Demo-Kapitel auskommentiert
werden.
%
\par{Nun viel Erfolg beim Schreiben der Arbeit!}
