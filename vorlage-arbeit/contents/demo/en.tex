% Copyright (C)  2013 Jana Traue (jana.traue[at]tu-cottbus.de)
%
% Permission is granted to copy, distribute and/or modify this document
% under the terms of the GNU Free Documentation License, Version 1.3
% or any later version published by the Free Software Foundation;
% with no Invariant Sections, no Front-Cover Texts, and no Back-Cover Texts.
% A copy of the license is included in the file entitled "LICENSE".
%
% =============================================================================
% LATEX chapter
% =============================================================================
%
% =============================================================================
% =============================================================================
\chapter{Demo Chapter (English)}
% =============================================================================
%
Welcome to the wonderful world of \LaTeX{}.
This document is a template for complex documents, like bachelor
or master thesis.
this context, complex does not refer to the content itself -
you are still responsible for it - but to the document structure.
In oder to benefit from this template, you should read the upcoming
sections at least once.
Eventually, the document tells how to get rid of this text in the final
version of the thesis.
%
% =============================================================================
\section{General Information about \LaTeX{}}
% =============================================================================
%
% -----------------------------------------------------------------------------
\subsection{Compiling}
% -----------------------------------------------------------------------------
%
In the top level folder is a file called \code{main.tex}.
Compile it via:
%
\begin{lstlisting}[language=bash]
pdflatex main.tex
\end{lstlisting}
%
Often, it is necessary to invoke this command twice in order to
get references right.
%
\par
%
The bibliography has to be translated separately, via:
%
\begin{lstlisting}[language=bash]
bibtex main
\end{lstlisting}
%
followed by another run of \code{pdflatex}.
%
\par
%
If you do not rename the file \code{main}, your advisers
may get documents called \code{main.pdf} from all their students
and will probably get confused.
Therefore, rename the file and include a unique identifier, like
your name.
%
\par
%
You do not need to type the above shown commands manually if you
use the provided Makefile.
A call of \code{make} invokes \code{pdflatex} as well as \code{bibtex}.
With the help of \code{make clean} you can get rid of the temporary files
that \LaTeX{} creates.
But be careful and take a look at the Makefile before you clean up.
It contains a list of suffixes and all files that match the pattern
are deleted.
In the worst case, one of your files is lost forever.
If you renamed the \code{main} file, you need to adjust the Makefile.
%
% -----------------------------------------------------------------------------
\subsection{Folder Structure}
% -----------------------------------------------------------------------------
%
Although it is possible to write your thesis in a single file,
it is not recommended, because you get lost easily.
This document's structure is a bit complex, but has proven to be useful for
a couple of years.
In addition to the main file, the root directory contains the following folders:
\begin{itemize}
	\item \code{bib}: bibliography files
	\item \code{contents}: tex files (the actual content)
	\item \code{contents/global}: global tex files which might be exchanged,
		like the title page. They do not contain content related data.	
	\item \code{figures}: figures (use a format which pdflatex supports)
\end{itemize}
%
%
% .............................................................................
\subsubsection{Latex Packages}
% .............................................................................
%
Due to the differences in existing operating systems and their distributions,
you have to figure out how to install \LaTeX{} on your own.
%
If you get an error message like
\begin{lstlisting}[language=bash]
! LaTeX Error: File '<package name>.sty' not found.
\end{lstlisting}
you are missing a \LaTeX{} package.
%
Sometimes you need packages that are not part of the standard \TeX distributions
our you are not allowed to install packages (e.g. in our lab).
In that case, you can download the package from
\href{http://www.ctan.org/}{CTAN} and copy them either to
\code{/usr/share/texmf/tex/latex/} (doesn't work in the lab)
or to the folder \code{/home/yourName/texmf/tex/} (that works).
%
Afterwards, run
\begin{lstlisting}[language=bash]
texhash .
\end{lstlisting}
in your home folder to register the package.
%
% -----------------------------------------------------------------------------
\subsection{Code Conventions}
% -----------------------------------------------------------------------------
%
\begin{itemize}
	\item Put each chapter into a different tex file or even folder.
        That avoids getting lost in long source files and forces
        you to structure the document.
	\item Mark sections and subsections within a chapter.
        It is easier to find a section if all of them are marked
        (like the used dashed lines in this file).
	\item Whenever code (like the name of a class or a folder's name)
		occurs, mark it with the \code{$\backslash$code\{\}} command
        \footnote{to change the style of these words later and for an easier 
            access of code sections within a document}.
		See also section \ref{demo:examples:listings}.
	\item Use meaningful labels. Start the name of a label with the kind of 
        element it refers to, like 
        \code{$\backslash$label\{chapter:introduction\}} or
		\code{$\backslash$label\{figure:introduction:section:name\}}. 
        This helps to avoid conflicts.
    \item Limit each line to 80 characters. If you ever change your editor
        or exchange the source code with a college,
        you do not need to waste time with adding new lines.
\end{itemize}
%
% -----------------------------------------------------------------------------
\subsection{Adjusting the Template}
% -----------------------------------------------------------------------------
%
The template is designed to be used for thesis in German as well as English.
When you use English, ignore the German parts (marked with \code{de}) 
in the source code.
At least, the following adjustments in {\LaTeX} files are necessary:
%
\paragraph{titlepage.tex}
%
Change the document's title in the pdf bookmark section.
Insert group and the according professor. 
The document type might be 'Bachelor Thesis'.
Complete title, your name, course of study and matriculation number.
You may add the names of your advisers and or remove this section.
Check whether 'adviser' is the correct term.
%
\paragraph{header.tex}
%
A description for common \href{http://www.komascript.de/}{KOMA-Script} options 
is provided at the beginning of the file.
Afterwards, there is a global switch for selecting English or German style - the
comment section.
Choose the language you want to exclude and include.
Some source files, like the title page, are already prepared for
multilingual use.
Be aware that there are no characters allowed on a line before
\begin{lstlisting}[language={[LaTeX]TeX}]
\begin{de}
\end{de}
\end{lstlisting}
That includes white spaces. A proper intending is impossible.
Violating this rule results in strange errors like
\begin{lstlisting}[language=bash]
Runaway argument?
! File ended while scanning use of \next.
<inserted text> 
                \par
\end{lstlisting}
Do not forget to run \code{make clean} after a language switch.

Fill in data for the hyperref package.
Hyphenations (at the end) tell {\LaTeX} how to split words.
%
%
\paragraph{acronyms.tex}
%
Acronyms are placed there. Name the longest acronym in order to get a correct 
layout.
Usually, at its first occurrence an acronym is given in the long form, 
like \ac{BTU}.
Thereafter, only the short form is used, like \ac{BTU}.
Both examples were invoked using \code{$\backslash$ac\{BTU\}} - {\LaTeX} does 
the rest.
You have to tell \LaTeX{} the longest acronym explicitly in order to align
the resulting table correctly.
The source files shows how to use plural versions.
%
% -----------------------------------------------------------------------------
\subsection{Using Tools}
% -----------------------------------------------------------------------------
%
\paragraph{Texmaker}
%
\href{http://www.xm1math.net/texmaker/}{Texmaker}
is available for several platforms. It supports compilation and viewing the 
document via short-cuts. 
For using it, adjust paths in its preferences section. 
Open the \code{main.tex} and set it as master file in the options menu. 
This has to be done every time texmaker is restarted. 
Thereafter, you may open any other chapter and start compiling the main
file by pressing \code{F6}. Bibtex is invoked by \code{F11}.
To view the pdf document hit \code{F7}.
In order to use the spell checker, provide a dictionary in the preferences 
section.
%
\paragraph{gVim}
was used for writing this document.
External commands are invoked via:
\begin{lstlisting}[language=bash]
:!<command>
\end{lstlisting}
Using this avoids a switch to the terminal.
A \code{:} followed by an arrow key allows navigation through recent commands.
If you do not know how to switch to the command mode, you probably should not 
use gVim at all.
%
\par
%
GVim includes a spell checker.
You can turn it on
\begin{lstlisting}[language=bash]
:set spell
\end{lstlisting}
and select a language
\begin{lstlisting}[language=bash]
:set spelllang=<lang>
\end{lstlisting}
Misspelled words are highlighted in the source code.
To get a list of suggestions for a misspelled word, place the cursor on it
and press \code{z=}.
If you are sure the word is correct, add it to the word list: \code{zg}.
The help file contains more information:
\begin{lstlisting}[language=bash]
:help spell
\end{lstlisting}
%
%
% \paragraph{Notes}
% %
% Handwritten notes are awful. They can not be included in revision control 
% systems.
% Paper might get lost. You might not be able to decipher the text.
% Therefore, convince your consultants to provide digital notes.
% Several free tools for annotating pdf documents exist. For Windows there is 
% \href{http://www.foxitsoftware.com/pdf/reader/}{Foxit Reader}. 
% The build-in Preview.app supports notes on Mac OS, but they are hard to read 
% using any other tool. 
% An alternative is \href{http://skim-app.sourceforge.net/}{Skim}.
% Still looking forward to find a Linux application besides 
% \href{http://okular.kde.org/}{Okular}. 
%
% -----------------------------------------------------------------------------
\subsection{Additional Hints}
% -----------------------------------------------------------------------------
%
\paragraph{Content versus Layout}
%
Try not to get a perfect layout while still writing the document. 
There will be at least one section which needs to be revised. 
And, of course, exactly this one is going to destroy the current layout. 
%
\paragraph{Check Logs}
%
{\LaTeX} creates a log file every time the document is compiled. 
Check it in order to find errors and warnings, like the already mentioned, 
overlong lines.
%
\paragraph{Use a Spell Checker!}
%
\par
%
As promised, I will tell you how to remove the demo chapter from your
document.
Open \code{main.tex} and turn the line with the demo chapter into a comment.
%
\par{Good Luck with your Thesis!}
