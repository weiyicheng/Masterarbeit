%%==========================================
\chapter{Zusammenfassung und Ausblick}
%%==========================================
%%==========================================
\section{Zusammenfassung}
%%==========================================
In dieser Arbeit, wird ein kultureller Vergleich zwischen Deutschland und China in Textileinzelhandel durch Sentiment Analyse von \acl{OCRs} gemacht. Durch die Recherche der vorherigen Forschungen wird ein Online Consumer Review in zwei Teile geteilt: quantitativen Teil und qualitativen Teil. Der quantitative Teil ist die Anzahl der Sterne, die die Kunden gegeben haben, und der qualitative Teil ist die schriftliche Beschreibung des Kundes. Durch Sentiment Analyse wird der qualitative Teil auch eine Polarität in Zahlen ausgedrückt. Diese Polarität beschreibt über das, wie positiv oder negativ der Kunde das Produkt gefühlt hat. Dadurch werden die \ac{OCRs} in dem quantitativen und qualitativen Aspekt drei statistische Attribute haben: Volumen, Valenz, und Varianz. Es besteht auch den Korrelationskoeffizient, der den Grad des Zusammenhang zwischen den quantitativen und qualitativen \ac{OCRs} beschreibt. Durch diese statistischen Attributen kann man die Gesamtsituation der \ac{OCRs} besser erkennen.

Basierend auf dem sechsdimensionalen Modell von \citeauthor{hofstede2013interkulturelle} kann man erfassen, dass Chinesen in der Kultur große Machtdistanz und geringe Unsicherheitsvermeidung haben, kollektivistisch sind, während die Deutschen geringe Machtdistanz und große Unsicherheitsvermeidung haben, individuell sind. Mit Hilfe der vorherigen Forschungen in dem kulturellen Bereich, wird es vorgeschlagen, dass die chinesischen \ac{OCRs} kürzer/weniger (kleineres Volumen), negativer (kleinere Valenz), und dichter (kleinere Varianz) in den beiden quantitativen und qualitativen Aspekten und unabhängiger (kleinerer Korrelationskoeffizient) zwischen den quantitativen und qualitativen Teilen als die deutschen \ac{OCRs} sind.

Mit diesen Hypothesen werden die \ac{OCRs} von den Untersuchungsobjekten aus den Plattformen des Online-Einzelhandels in den beiden Ländern gesammelt. In dieser Arbeit werden vier unterschiedliche Sportkleidungen als Untersuchungeobjekte ausgewählt. Die chinesischen \ac{OCRs} werden durch Google Translate in Deutsch übersetzt. Nach der Textverarbeitung werden die gesammelten \ac{OCRs} durch Sentiment Analyse durchgeführt, damit jedes Review eine Polarität über den Grad des Gefühls von dem Kunden hat. 

Nach der Prüfung der Hypothesen, wird das gefunden, dass die Kultur unterschiedlichen Einflüssen auf den \ac{OCRs} in dem quantitativen und qualitativen Aspekt hat. In dem quantitativen Aspekt, sind die chinesischen \ac{OCRs} dichter als die deutschen, wie die Theorien gezeigt, aber die chinesischen \ac{OCRs} sind mehr und positiver als die deutschen \ac{OCRs}. In dem qualitativen Aspekt, sind die chinesischen \ac{OCRs} kürzer, negativer, und dichter als die deutschen, genau wie die vorherigen Forschungen. Die Zusammenhänge in den beiden Ländern sind positiv, und der chinesische Zusammenhang zwischen den quantitativen und qualitativen \ac{OCRs} ist schwächer als den deutschen.

Die Gründe, warum es solche Unterschiede in der Theorie und Tat gibt, werden in Kapitel \ref{diskussion} diskutiert. Zusammenfassend werden die quantitativen \ac{OCRs} einfacher als die qualitativen Teilen durch die Kultur beeinflusst, besonders in China. Die kollektivistische Kultur von den chinesischen Kunden motiviert sie, mehrere aber kürzere \ac{OCRs} zu schreiben und beeinflusst groß auf die quantitativen \ac{OCRs}, damit die chinesischen Kunden mehr ``5 Sterne'' geben, obwohl sie nicht zufrieden sind. Deshalb sind die chinesischen quantitativen \ac{OCRs} viel positiver als die deutschen und gleichzeitig ist der Zusammenhang zwischen chinesischen quantitativen und qualitativen \ac{OCRs} schwächer. 

Mit dieser Kenntnisse sollten die Kunden beachten, dass die chinesischen Sterne (also die quantitativen \ac{OCRs}) kein glaubwürdiges Kennzeichen in der Kaufentscheidung sind, während die deutschen Sterne ein Verweis sind. Die Kunden in den beiden Ländern sollten auch mehr die schriftlichen Teilen der \ac{OCRs} durchlesen, um die Glaubwürdigkeit zu erhöhen und mehrere Informationen über die Attribute des Produkts herauszufinden, damit sie richtige Kaufentscheidung treffen könnten.

Die Unternehmen des Produkts sollten auch die schriftlichen Teilen analysieren, um die Stärke und Schwäche des Produkts durch die Rezensionen der Kunden kennenzulernen, und die Probleme schnell reagieren. Durch diese Analyse könnten sie die Kunden fokussierenden Attributen erkennen und verbessern, damit sie besseres Produkt oder Service anbieten.

Die Betreiber der Plattformen sollten diese Ergebnisse beachten. Sie sollten die \ac{OCRs} möglicherweise gut in einer bestimmt Reihenfolge (aber nicht zeitlich) ordnen, damit diese nutzlosen \ac{OCRs} nicht direkt an den ersten Positionen gestellt werden, zum Beispiel: die zu kurzen \ac{OCRs}. Aber die hilfreichen Kundenrezensionen sollten nach vorne gestellt werden, damit die anderen Kunden schnell die nützlichen Informationen finden könnten. Wenn die Plattformen die Kunden fokussierenden Aspekte oder die Attituden der Kunden aus den \ac{OCRs} automatisch zusammenfassen, wäre es natürlich besser.
%%==========================================
\section{Limitationen}
%%==========================================
Die in dieser Forschung ausgegebenen Ergebnisse sollten in Verbindung mit verschiedenen Limitationen berücksichtigt werden. Diese Limitationen betrachten aus einer Vielzahl von Blickwinkeln:
\begin{enumerate}
	\item \textbf{Wirtschaftliche Limitationen}: 
	\begin{enumerate}
        	\item Diese Studie ist im Bereich des Textileinzelhandels. Das bedeutet, dass die Ergebnisse dieser Arbeit nur über den Textileinzelhandel sind. Es braucht noch nachzuweisen, dass die Ergebnisse domainspezifisch oder nicht sind. Und diese Untersuchungsobjekte sind die Sportkleidungen, die nicht alle Güter im Textileinzelhandel repräsentieren könnten.
        	\item Die in dieser Arbeit genutzte Maßnahmen sind statistisch. Diese Maßnahmen benötigen zahlreiche \ac{OCRs} von gleichen Produkten in den beiden Ländern zu analysieren. Und nur nach Kauf des Produkts könnten die Kunden \ac{OCRs} geben, deshalb sind die Untersuchungsobjekte beschränkt, weil sie in China und Deutschland beiden sehr gut verkauft werden müssen. Deshalb sind die \ac{OCRs} meistens positiv. 
        \end{enumerate} 
	\item \textbf{Limitationen von der Sentiment Analyse}:
	\begin{enumerate}
		\item Die in dieser Arbeit verwendete Sentiment Analyse ist nur im Dokument-Level. Es gibt Aspekt-Level in Sentiment Analyse. Und in diesem Level fokussiert die Analyse über die konkrete Aspekte der Kunden, zum Beispiel: wie gut ist die Kleidung in Ästhetik, Performance oder äußeren Attributen in den \ac{OCRs}? Diese Attitude von Kunden über die Attribute des Produkts werden in dieser Arbeit nicht studiert.
		\item Der Lexikon sollte verbessert wird. Und der Algorithmus betrachtet keine Phrase, keinen Komparativ und Superlativ. Die Phrase ``nicht gut'' ist natürlich kein positiver Ausdruck, und die Wörter ``besser'' und ``best'' sollten auch nicht gleiche Polarität wie ``gut'' haben. Aber in diesem Algorithmus werden diese Situationen ignoriert. 
	\end{enumerate}
	\item \textbf{Limitationen von Übersetzung}:
	\begin{enumerate}
		\item Der in dieser Arbeit gemachte Vergleich ist kulturübergreifend und mehrsprachig, deshalb benötigt man eine Sprache in andere zu übersetzen. Wegen der großen Menge der Rohdaten, wird eine Übersetzungsmaschine hier verwendet. Obwohl die Einflüsse klein sind, wie die Diskussion im Abschnitt \ref{mehrsprachig} gezeigt wird, gibt es wirklich Einflüsse.
		\item Weil China und Deutschland unterschiedlichen Kulturen haben, haben die Wörter unterschiedliche Bedeutung. Zum Beispiel: das Wort ``Stolz''in Deutschland ist positiv aber in China ist es negativ, trotz das Wort gleich in beiden Ländern ist.
	\end{enumerate}
	\item \textbf{Limitationen von Statistik}:
	\begin{enumerate}
		\item In dieser Arbeit, wird es einige Ergebnisse im Allgemeinen herausgefunden, aber es gibt auch Ausnahme. Die Ausnahme ist von den deutschen Daten aus Puma. Der Grund dieser Außerordentlichkeit ist noch nicht gefunden.  
	\end{enumerate}
\end{enumerate}
%%==========================================
\section{Zukünftige Forschung}
%%==========================================
In Zukunft, werden die weiteren Forschungen erst mehrere Untersuchungsobjekte im Textileinzelhandel durchgeführt, um die Ergebnisse in dieser Arbeit zu beweisen. Wenn man mehrere Untersuchungsobjekte analysieren würde, könnte man auch erkennen, ob die obig genannte Ausnahme nur ein seltener Fall ist. Und man sollte zwischen mehreren Ländern den kulturellen Vergleich durchführen, damit die kulturellen Einflüsse von den Dimensionen auf die \acl{OCRs} besser erkennt werden könnten. 

Für die weiteren Forschungen, ist es auch wichtig, die Limitationen von der Sentiment Analyse zu lösen. Man kann die Analyse im Aspekt-Level durchführen und herausfinden, welche Unterschiede es in der konkreten Aspekten der Kunden zwischen Kulturen gibt, und welche Zusammenhänge es zwischen den Unterschieden und kulturellen Dimensionen von \citeauthor{hofstede2013interkulturelle} gibt. Und der Lexikon und Algorithmus sollten auch in Zukunft verbessert werden, um richtigere Ergebnisse zu bekommen.

Die kulturellen Gründe werden vorgeschlagen, aber noch nicht durch Studien bewiesen. Mit den Ergebnissen dieser Arbeit kann man weitere Forschungen über die \acl{OCRs} betreiben, insbesondere eine Ursachenforschung durchführen, damit man die Zusammenhänge zwischen den kulturellen Dimensionen und den Ergebnissen bestimmen kann.

Die Motive und Wirkungen dieser Unterschiede haben auch in dieser Arbeit in der Konsumentensicht, Unternehmensicht und der Sicht der Plattformsbetreiber diskutiert geworden. Aber es braucht auch die obig dargestellten Motive und Wirkungen durch weitere Forschungen zu bestimmen.

In Zusammenfassung, macht diese Arbeit einen kulturellen Vergleich von den \acl{OCRs} im Textileinzelhandel zwischen Deutschland und China durch Sentiment Analyse, und einige Unterschiede zwischen den beiden Ländern werden in dieser Arbeit herausgefunden. Die deutschen \acl{OCRs} sind qualitativer und die chinesischen \acl{OCRs} sind quantitativer, und in dem quantitativen Aspekt sind die chinesischen \acl{OCRs} positiver aber in dem qualitativen Aspekt ist die Situation geändert. Die deutschen \acl{OCRs} sind dezentralisiert und die chinesischen \acl{OCRs} sind dicht zentralisiert. Die deutschen Kunden geben die quantitativen \acl{OCRs} abhängiger von dem schriftlichen Teil, als die chinesischen Kunden. Es ist erwünscht, dass diese Ergebnisse und Diskussionen für die Konsumenten, Unternehmen der Produkten, und die Betreiber der Plattformen im Textileinzelhandel nützlich sind.
