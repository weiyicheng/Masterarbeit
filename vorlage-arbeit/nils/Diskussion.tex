% ===========================================================
\chapter{Diskussion} \label{diskussion}
% ===========================================================
Es gibt schon viele Forschungen über die \acl{OCRs}, und der Begriff, die Motivationen und die Auswirkungen werden durch die Studien breits reichlich studiert. Aber es gibt nicht viele Ergebnisse über die kulturellen Einflüsse. Gibt es Unterschied in den \ac{OCRs} aufgrund der kulturellen Einflüsse? Wenn ja, welche Unterschiede, und warum? Diese Fragen sind noch zu beantworten. Besonders über die qualitativen \ac{OCRs}, also dem schriftliche Teil, gibt es noch wenige Studien. In dieser Arbeit werden die \ac{OCRs} nach \citet{Shrihari2012} in den quantitativen und qualitativen Teil getrennt und jedem Teil werden drei Attribute gegeben: Volumen, Valenz, und Varianz. Außerdem haben die beiden Teile ein gemeinsames Attribut: Der Korrelationskoeffizient. Durch dieses digitale Attribut kann man die statistischen Maßnahmen benutzen, um die allgemeinen Unterschiede hinter den Rohdaten herauszufinden.

Für diese Attribute werden vier Hypothesen gebildet, um die Unterschiede zwischen China und Deutschland zu entdecken. Hypothese \ref{hyp:1}, \ref{hyp:2} und \ref{hyp:3} beziehen sich auf das Volumen, die Valenz, und die Varianz der \ac{OCRs} sowohl in dem quantitativen Aspekt als auch in dem qualitativen Aspekt. Hypothese \ref{hyp:4} bezieht sich auf den Zusammenhang der quantitativen und qualitativen \ac{OCRs}. Diese vier Hypothesen versuchen, durch die Attribute die statistischen Unterschiede herauszufinden. Tabelle \ref{tab:ergebnis_hypothese} zeigt die zusammenfassenden Ergebnisse.

\begin{table}[h]
\centering
\resizebox{\textwidth}{!}{%
\begin{tabular}{|c|c|c|c|}
\hline
Hypothese & Ergebnisse              & Quantitativer Aspekt           & Qualitativer Aspekt         \\ \hline
\ref{hyp:1}         & Volumen                 & China \textgreater Deutschland & China \textless Deutschland \\ \hline
\ref{hyp:2}          & Valenz                  & China \textgreater Deutschland & China \textless Deutschland \\ \hline
\ref{hyp:3}          & Varianz                 & China \textless Deutschland*   & China \textless Deutschland \\ \hline
\ref{hyp:4}          & Korrelationskoeffizient & \multicolumn{2}{c|}{China \textless Deutschland*}            \\ \hline
\end{tabular}
}
\caption[Die zusammenfassenden Ergebnisse von Hypothesen]{Die zusammenfassenden Ergebnisse von Hypothesen (*: Ausnahme ist Puma) (Quelle: Eigene Darstellung)}
\label{tab:ergebnis_hypothese}
\end{table}


In den folgenden Teilen wird versucht, die kulturellen Gründen für diese Ergebnisse möglicherweise nach den theoretischen Grundlagen herauszufinden. Des weiteren wird diskutiert, welche Auswirkungen es auf den Kunden \ac{bzw.} das Unternehmen gibt.
%%===========================================================
\section{Qualitativ gegenüber Quantitativ}
%%===========================================================
Laut \citet{Lam2009} sind die Menschen zurückhaltend bei dem Informationsaustausch, wenn sie in einer Kultur mit großer Machtdistanz leben, und nach \citet{Lam2009, liu2001relationships, dawar1996cross, money1998explorations} können die Menschen mit einer hohen Unsicherheitsvermeidung ein höheres Niveau des Meinungsaustauschs sowie der Meinungssucht. In Abbildung \ref{fig:sechsdimensionen} sieht man schon, dass es in China eine sehr große Machtdistanz gibt, und die deutsche Unsicherheitsvermeidung ist höher als in China. Deshalb wird in Hypothese \ref{hyp:1} erwartet, dass das chinesische Volumen kleiner als das von den Deutschen ist, sowohl bei den quantitativen als auch bei den qualitativen \ac{OCRs}. Diese Hypothese meint, dass die deutschen Kunden länger und mehr schreiben würden während die chinesischen Kunden kürzer und weniger schreiben würden.

Aber die Wirklichkeit sieht anders aus. Es stimmt, dass die Deutschen länger schreiben und die Chinesen kürzer schreiben. Aber die Chinesen schreiben viel lieber als die Deutschen. Die Menge der chinesischen \ac{OCRs} ist das 15fache der deutschen \ac{OCRs} bei den Untersuchungsobjekten in dieser Arbeit. In dieser Situation wird zusammengefasst, dass die Deutschen qualitativ schreiben wollen aber die chinesischen \ac{OCRs} sind quantitativ.

Es ist klar, dass wer zurückhaltend beim Informationsaustausch ist, kürzer schreiben wird. Diese Zurückbehaltung begründet sich in der großen Machtdistanz in der chinesischen Kultur nach \citet{Lam2009}. Die Deutschen, die mit kleiner Machtdistanz in ihrer Kultur leben, sind offener als die Chinesen und vermeiden die Unsicherheit lieber \citep{Lam2009, liu2001relationships, dawar1996cross, money1998explorations}, daher wollen sie mehr Information austauschen, deswegen schreiben sie länger.

Aber die Frage, warum das chinesische quantitative Volumen so groß ist, steht noch in dem theoretischen Bereich offen. Es sollte mindestens eine extra Motivation der Chinesen für das Schreiben einer Review im Vergleich zu der Deutschen geben. Es wird vorgeschlagen, dass diese Motivation aus der kollektivistischen Kultur kommt. Bei der kollektivistischen Kultur ist die Einstellung ``alle machen, dann mache ich auch'' normal. Es entsteht die Motivation, eine Review für das Produkt zu schreiben, wenn das Produkt schon viele Reviews hat. Aber die Motivation ist nicht stark genug für eine lange Review. Die Chinesen möchten zeigen, dass sie noch in der Gruppe sind, in der alle Reviews schreiben. Diese in-Gruppenmitgliedschaften ist für sie wichtig. Aber die Deutschen, als die Mitglieder einer individualistischen Kultur, sind unabhängig von anderen, und betrachten die Getrenntheit und Entfernung von in-Gruppen. \citep{singelis1994measurement}

Für die anderen chinesischen Konsumenten, sind die chinesischen \ac{OCRs} schlechter als die Deutschen, weil die Kunden, die \ac{OCRs} geschrieben haben, wenige Information gegeben haben, obwohl die Menge groß ist. Wenn man nur einfach ``gut'' oder ``Okay'' geschrieben hat, wissen die anderen Leute nicht, worauf sich die Review bezieht. Deshalb müssen sie noch mehr lesen, um die richtigen und wichtigen Informationen zu finden. Die kurzen \ac{OCRs} werden als ``Noise Text'' von den anderen Kunden oder potentiellen Kunden genannt. In dieser Hinsicht sollte der Kunde auch wissen, dass es keine Hilfe ist, wenn er oder sie nur ein kurzes Review geschrieben hat. Der Kunde sollte vermeiden, eine kurze Review zu schreiben, sondern möglicherweise die Attribute des Produkts beschreiben, damit er oder sie den anderen helfen kann. 

Andererseits gibt es diese Situation für die deutschen \ac{OCRs} kaum. Aber bei den deutschen \ac{OCRs} gibt es noch andere Probleme. Zu lange Reviews sind auch sinnlos, weil die anderen Menschen keine Zeit haben, um die lange Review durchzulesen. Deshalb könnte die lange Review nicht so viel helfen. Die Review sollte nicht zu kurz oder zu lang sein, aber die Attribute des Produkts sollten als Stichpunkte beschrieben werden. Zum Beispiel, ist es eine gute Review, wenn man über die Vorteile und Nachteile der Kleidung in den Attributen ``Ästhetik'', ``Performance'' und ``äußere Attribute'' geschrieben hat.

Es war auch eine Herausforderung, die die Unternehmen oder die Betreiber der Plattformen meistern müssen, die ``Noise Text'' zu vermeiden oder wenigstens sich nicht an der ersten Stelle stehen zu lassen. Beim Vermeiden kann man einfach die minimale Anzahl der Wörter beschränken, aber wie gesagt, die Motivation ist nicht so stark, damit kann man einfach das Schreiben der Review aufgeben. Deshalb ordnen die Unternehmen gerne die \ac{OCRs} nochmal, damit sie nicht in zeitlicher Reihenfolge, sondern in nützlicher Reihenfolge sind. 

Die chinesische Plattform ``Tmall'' ordnet die \ac{OCRs} so, dass man die längere \ac{OCRs} mit Fotos zuerst lesen kann. Die deutsche Plattform ``Amazon'' ordnet die \ac{OCRs} durch ``Hilfreich''. Wenn die anderen Leute denken, dass dieser Kommentar für sie hilfreich ist, drücken sie ``Ja'' und die anderen sehen, dass dieser Kommentar einem Mensch hilft. Der Kommentar kann vorne stehen, damit mehrere Menschen ihn lesen können.
%%===========================================================
\section{Positiv gegenüber Neutral} \label{sec:diskussion:2}
%%===========================================================
Nach \citet{liu2001relationships} geben die Personen aus geringerer Unsicherheitsvermeidung häufiger negative Kommentare, als die Personen in Kulturen mit höherer Unsicherheitsvermeidung, wenn sie schlecht bedient werden. Und laut \citet{donthu1998cultural}, geben die Menschen in Kulturen mit hoher Unsicherheitsvermeidung im Allgemeinen positivere Kommentare, als die mit niedriger Unsicherheitsvermeidung. Und Wie aus Abbildung \ref{fig:sechsdimensionen} ersichtlich ist, sind die Unsicherheitsvermeidungen der Deutschen relativ hoch im Vergleich zu den Chinesen. Deshalb wird in Hypothese \ref{hyp:2} vorgeschlagen, dass die deutschen Valenzen größer als die chinesischen Valenzen der \ac{OCRs} in dem quantitativen und auch in dem qualitativen Aspekt sein sollten. Also, die deutschen \ac{OCRs} sollten positiver als die chinesischen \ac{OCRs} sein.

Aber es stimmt nur teilweise in der Praxis. Die deutschen qualitativen \ac{OCRs} sind tatsächlich positiver als die chinesischen durch die Sentiment Analyse, aber in dem quantitativen Aspekt präsentieren sich die chinesischen \ac{OCRs} positiver. Im Durchschnitt geben die Deutschen für die Untersuchungsobjekte nur 4,51 Sterne im Vergleich zu 4,88 Sterne von den Chinesen (Maximal 5 Sterne). Die durchschnittliche Valenz von den deutschen qualitativen \ac{OCRs} ist 0,6702, schon relativ hoch (beim einzelnen Wort bis 1, und in dem Lexikon von \citet{Remus2010} ist 0,6702 zwischen ``zuvorkommend, 0,6669'' und ``romantisch, 0,6965''). Die chinesische durchschnittliche Valenz ist 0,2525 und es ist im Lexikon von \citet{Remus2010} zwischen ``fröhlich, 0,2501'' und ``gefallen, 0,2578''. Daher sind die deutschen \ac{OCRs} positiv in dem qualitativen Aspekt und die chinesischen qualitativen \ac{OCRs} sind neutral.

Die kulturelle Begründung der positiveren deutschen qualitativen \ac{OCRs} ist nach den Forschungen von \citet{donthu1998cultural} klar, und laut \citet{liu2001relationships} sollen die chinesischen \ac{OCRs} negativer als die deutschen sein, aber dieses Phänomen, dass die chinesischen Kunden viel lieber 5 Sterne geben (siehe in Abbildung \ref{fig:struktur_allgemein}), wird noch nicht begründet. Möglicherweise gab es einen wirtschaftliche Anreiz für die chinesischen Kunden, die 5 Sterne zu geben. Wenn man in den kulturellen Dimensionen denkt, wird vermutet, dass die kollektivistische Kultur der Chinesen das Phänomen erzeugt. Nach \citet[p. ~65]{hofstede1998masculinity}, werden die individuellen Entscheidungen (zum Beispiel: wie viele Sterne gebe ich) in kollektivistischer Kultur im Konsens mit der Gruppe gemacht und es gibt keine rein individuellen Entscheidungen zu treffen. Aber im Gegensatz dazu denken die Menschen in individualistischen Kulturen eher selbst als autonom zu denken und stellen die individuellen Interessen an erster Stelle \citep{shweder1990defense}. 

Das heißt, dass die chinesische einzelne Valenz von den quantitativen Reviews (also Sterne) von den bereits existierten \ac{OCRs} groß beeinflusst wird, während die einzelne deutsche Valenz individuell ist. Wenn es schon viele ``5 Sterne'' \ac{OCRs} gab, könnte es sein, dass die chinesischen Kunden ``5 Sterne'' geben wollen, obwohl sie nicht zufrieden sind. Nach der Einsicht in die Rohdaten, wird das Beispiel gefunden: ``
\begin{CJK*}{UTF8}{gbsn}
	老公说质量不怎么样。
\end{CJK*}
'' Die Kundin meint, dass ihr Mann sagt, dass die Qualität des T-Shirts nicht so gut ist, trotzdem hat sie ``5 Sterne'' gegeben. Es ist nur ein Beispiel, aber es gibt noch einige ähnlichen Situationen in den chinesischen Daten.

Die chinesischen Kunden sollten vorsichtiger sein, als die Deutschen, weil die anderen chinesischen Kunden ``5 Sterne'' gegeben haben könnten, obwohl das Produkt nicht gut ist. Es ist nicht genug nur die Sterne zu lesen, man muss auch, besonders bei den chinesischen Kunden, den Text dazu lesen.

Für die Unternehmen, ist es wichtig zu wissen, dass es Probleme geben könnte, obwohl die quantitativen Valenzen gut scheinen. Sie sollten auch die Texte durchlesen, und schnell reagieren, um die Probleme zu beseitigen. 

\citet{Luo2014} haben herausgefunden, dass durch die stärkere Wirkung der zweiseitigen Informationen (im Vergleich zu einseitigen Informationen) die Wahrnehmung von Glaubwürdigkeit der Informationen der \ac{eWOM} Leser gestärkt wird, wenn die Leser aus individualistischen Kulturen sind, verglichen mit denen, die aus kollektivistisch kulturellen Orientierung sind. Deshalb sollte die Reihenfolge der \ac{OCRs} in Deutschland und China unterschiedlich sein. Die Betreiber von Amazon sollten nicht nur die positiven Bewertungen sondern auch einige negativen Bewertungen an die ersten Positionen stellen, damit für die deutschen Kunden die Glaubwürdigkeit der \ac{OCRs} höher ist. Die Betreiber von Tmall brauchen das vielleicht nicht so zu machen.
%%===========================================================
\section{Dezentralisiert gegenüber Zentralisiert}
%%===========================================================
Laut \citet{hofstede2013interkulturelle} ist die Ungleichheit bei großer Machtdistanz erwünscht und wird erwartet, deswegen sollten die \ac{OCRs} dezentralisiert sein. Aber nach der Forschung der \citeauthor{Luo2014} sollten die \ac{OCRs} konsistenter und dichter beim Kollektivismus sein. Die Chinesen leben in der kollektivistischen Kultur mit großer Machtdistanz. Die Deutschen leben in individueller Kultur mit geringer Machtdistanz. Weil die vorherigen Forscher die Dimension ``Individualismus gegenüber Kollektivismus'' als eine Tiefenstruktur denken \citep{grennfield2000approaches,sia2009web, triandis2001individualism}, wird in Hypothese \ref{hyp:3} vorgeschlagen, dass die chinesischen \ac{OCRs} zentralisiert und die deutschen \ac{OCRs} dezentralisiert sein sollten.

Die Tatsache stützt die Hypothese. In dem quantitativen und auch qualitativen Aspekt, sind die Varianzen von den chinesischen \ac{OCRs} kleiner als die von den deutschen im Allgemeinen. Für das jeweilige Produkt, ist die Situation fast gleich, aber es gibt eine Ausnahme bei den quantitativen \ac{OCRs} von Puma. Der Grund für diese Außerordentlichkeit ist noch nicht klar. Vielleicht ist es aufgrund der Machtdistanz nach \citet{hofstede2013interkulturelle}, aber dafür ist nicht der Schatten eines Beweises zu erbringen. Durch diese Ergebnisse, ist es auch ersichtlich, dass die Dimension ``Individualismus gegenüber Kollektivismus'' eine Tiefenstruktur und wichtigste Dimension ist, wie die Forscher schon vorher gemacht haben. 

Die chinesischen Kunden sollten vor dem Lesen der \ac{OCRs} wissen, dass sie viel dichter sein werden. Damit können sich die Kunden nicht einfach von den guten \ac{OCRs} bei der Kaufentscheidungen beeinflussen lassen. Man sollte mehr lesen oder andere Benutzererfahrungen suchen um die Schwächen und Stärken des Produkts herauszufinden und damit die Informationen eher umfassend sind. Die deutschen Kunden sollten auch wissen, dass die \ac{OCRs} individuell und dezentralisiert sind. Das heißt, dass die \ac{OCRs} nicht nur positiv sein können. Individuelle Erfahrung oder Wahrnehmung könnten nicht selbst anpassend sein, wie zum Beispiel das ein Kunde schreibt ``Die Farbe mag ich nicht'' und deswegen nur einen Stern gibt. 

Die Unternehmen sollten auch diese Verteilung erkennen. In China sind die \ac{OCRs} zentralisiert und dicht. Die Unternehmen versuchen, das Zentrum der \ac{OCRs} möglicherweise zur positiven Seite zu entwickeln. Die wichtigen Schwächen werden durch die \ac{OCRs} häufig gezeigt und diese Probleme sollten die Unternehmen zuerst lösen. In Deutschland sind die \ac{OCRs} dezentralisiert, deswegen müssen die Unternehmen sie selbst zusammenfassen und die Stärken und Schwächen kennen, um bessere Produkte bieten zu können.

Wenn die Betreiber der Plattformen die Technologien anbieten könnten, den Kunden fokussierende Aspekte oder die Attitüden der Kunden aus den \ac{OCRs} automatisch zusammenzufassen, wäre es besser. Damit die Kunden und Unternehmen nicht so viel lesen müssen und einfach erkennen worin die Stärken und Schwächen des Produkts liegen. Tmall aus China hat diese Zusammenfassung schon realisiert, aber bei Amazon gibt es diese noch nicht.
%%===========================================================
\section{Abhängig gegenüber Unabhängig}
%%===========================================================
Obwohl es schon viele Forschungen über die \ac{OCRs} gibt, bleibt der Zusammenhang zwischen den quantitativen und qualitativen \ac{OCRs} heutzutage nicht in der Sicht der Wissenschaftler. Gibt es bestimmten Zusammenhänge? Wenn ja, ist der Zusammenhang positiv oder negativ? Gibt es kulturelle Unterschiede in dem Zusammenhang? Diese Fragen werden noch nicht beantwortet. Hypothese \ref{hyp:4} schlägt vor, dass es den Zusammenhang gibt, und der deutsche Zusammenhang zwischen quantitativen und qualitativen \ac{OCRs} ist stärker als der von den von chinesischen \ac{OCRs}. Dieser Zusammenhang wird durch den Korrelationskoeffizienten und Rangkorrelationskoeffizienten gemessen.

Durch die Tests wird die Hypothese \ref{hyp:4} in den meisten Fällen gestützt. Es gibt positive Zusammenhänge zwischen den quantitativen und qualitativen \ac{OCRs} und der Zusammenhang von den deutschen \ac{OCRs} ist stärker als der von den chinesischen \ac{OCRs}. Diese Ergebnisse werden von den Ergebnissen der Hypothese \ref{hyp:2} gestützt, weil die chinesische quantitative Valenz größer als die deutsche ist. Jedoch ist die deutsche Valenz in dem qualitativen Aspekt größer. Wie es in Abschnitt \ref{sec:diskussion:2} durch das Beispiel schon gezeigt wird, sind die chinesischen quantitativen \ac{OCRs} manchmal unabhängig von den qualitativen \ac{OCRs}, aber die Deutschen geben die Sterne in Abhängigkeit davon, was sie schreiben wollen.

Es gibt in den Ergebnissen auch eine Ausnahme von Puma. Der deutsche Zusammenhang ist nicht mehr positiv sondern negativ. Gemäß des Korrelationskoeffizient ist dieser Zusammenhang fast Null. Der Grund der Außerordentlichkeit ist noch nicht bekannt.

Die Frage, warum die Chinesen ohne oder mit kleinerer Abhängigkeit Sterne geben und \ac{OCRs} schreiben, ist noch offen. Vielleicht gibt es wirtschaftliche Anreize für die Kunden, die ``5 Sterne'' zu vergeben. In dem kulturellen Aspekt, werden die Sterne von chinesischen Kunden von den bereits existierten \ac{OCRs} größer beeinflusst, als die Deutschen, wegen des Kollektivismus. 

Die Kunden schreiben die qualitativen \ac{OCRs} (Text), um die wirkliche Situation zu beschreiben und das eigene Gefühl auszudrücken. Je mehr man Text schreibt, desto mehr Informationen werden über die Attribute des Produkts beschrieben. Zum Beispiel: Die allgemeinen Attributkategorien für Kleidung sind Ästhetik, Performance, und äußere Attribute. Die qualitativen \ac{OCRs} werden meistens über diese drei Attributkategorien geschrieben. Wie sich bereits aus Abbildung \ref{fig:top10} ersehen lässt, schreiben die deutschen Kunden mehr über die Performance (``gross'', ``angenehm'', ``passt'', ``trage'' in Top 10 Wörtern) und die chinesischen Kunden schreiben mehr über die Performance (``tragbaren'' und ``gross'') und die äußere Attribute (``schnell'', ``echt'' und ``logistics'' in Top 10 Wörtern). Diese Informationen sind individuell und können nicht einfach von anderen Leuten beeinflusst werden. In diesem Sinne sind die qualitativen \ac{OCRs} glaubwürdiger als die quantitativen \ac{OCRs}. 

Diese Ergebnisse sind wichtig für die Kunden und die Unternehmen. Die chinesischen Kunden sollten erkennen, dass die Sterne nicht so glaubwürdig sind, weil sie einfach beeinflusst werden könnten. Mehr zu lesen und die Attribute über die Kleidung zusammenzufassen ist besser als an die Sterne zu Glauben, weil der Zusammenhang nicht stark ist. Für die Deutschen ist das Problem nicht so groß wegen des stärkeren Zusammenhang, aber für die Sicherheit ist es auch besser, die qualitativen Teile der \ac{OCRs} durchzulesen.

Die Unternehmen sollten auch nicht durch die Mehrheit der ``5 Sterne'' verwirrt sein. Sie sollten die qualitativen \ac{OCRs} durchlesen und auf die Probleme schnell reagieren, damit sie bessere Produkte und Service anbieten können. Und wie vorher schon gesagt, wird es besser sein, dass die Betreiber der Plattform die Webseite so organisieren, das die von Kunden fokussierenden Attribute automatisch zusammengefasst sind.
