%%==========================================
\chapter{Zusammenfassung und Ausblick}
%%==========================================
%%==========================================
\section{Zusammenfassung}
%%==========================================
In dieser Arbeit, wird ein kultureller Vergleich zwischen Deutschland und China in Textileinzelhandel durch Sentiment Analyse von \acl{OCRs} gemacht. Durch die Recherche der vorherigen Forschungen wird ein Online Consumer Review in zwei Teile geteilt: den quantitativen Teil und den qualitativen Teil. Der quantitative Teil ist die Anzahl der Sterne, die die Kunden gegeben haben, und der qualitative Teil ist die schriftliche Beschreibung des Kunden. Durch Sentiment Analyse wird der qualitative Teil auch durch eine Polarität in Zahlen ausgedrückt. Diese Polarität beschreibt, wie positiv oder negativ der Kunde das Produkt empfunden hat. Dadurch werden die \ac{OCRs} in dem quantitativen und qualitativen Aspekt drei statistische Attribute haben: Volumen, Valenz, und Varianz. Es besteht auch ein Korrelationskoeffizient, der den Grad des Zusammenhangs zwischen den quantitativen und qualitativen \ac{OCRs} beschreibt. Durch diese statistischen Attributen kann man die Gesamtsituation der \ac{OCRs} besser erkennen.

Basierend auf dem sechsdimensionalen Modell von \citeauthor{hofstede2013interkulturelle} kann man erfassen, dass Chinesen in der Kultur große Machtdistanz und geringe Unsicherheitsvermeidung haben, kollektivistisch sind, während die Deutschen geringe Machtdistanz und große Unsicherheitsvermeidung haben, individuell sind. Mit Hilfe der vorherigen Forschungen in dem kulturellen Bereich, wird vorgeschlagen, dass die chinesischen \ac{OCRs} kürzer/weniger (kleineres Volumen), negativer (kleinere Valenz), und dichter (kleinere Varianz) in den beiden quantitativen und qualitativen Aspekten und unabhängiger (kleinerer Korrelationskoeffizient) zwischen den quantitativen und qualitativen Teilen als die deutschen \ac{OCRs} sind.

Mit diesen Hypothesen werden die \ac{OCRs} von den Untersuchungsobjekten aus den Plattformen des Online-Einzelhandels in den beiden Ländern gesammelt. In dieser Arbeit werden vier unterschiedliche Sportkleidungen als Untersuchungsobjekte ausgewählt. Die chinesischen \ac{OCRs} werden durch Google Translate in Deutsch übersetzt. Nach der Textverarbeitung werden die gesammelten \ac{OCRs} durch Sentiment Analyse durchgeführt, damit jede Review eine Polarität über den Grad des Gefühls des Kunden hat. 

Nach der Prüfung der Hypothesen, wird das gefunden, was die unterschiedlichen kulturellen  Einflüsse auf die \ac{OCRs} in quantitativen und qualitativen Aspekten ausmacht. Im quantitativen Aspekt, sind die chinesischen \ac{OCRs} dichter als die deutschen, wie die Theorien gezeigt haben, aber die chinesischen \ac{OCRs} sind mehr und positiver als die deutschen \ac{OCRs}. In dem qualitativen Aspekt, sind die chinesischen \ac{OCRs} kürzer, negativer, und dichter als die deutschen, genau wie die vorherigen Forschungen gezeigt haben. Die Zusammenhänge in den beiden Ländern sind positiv, und der chinesische Zusammenhang zwischen den quantitativen und qualitativen \ac{OCRs} ist schwächer als der von den deutschen.

Die Gründe, warum es solche Unterschiede in der Theorie und der Praxis gibt, werden in Kapitel \ref{diskussion} diskutiert. Zusammenfassend werden die quantitativen \ac{OCRs} stärker als die qualitativen durch die Kultur beeinflusst, besonders in China. Die kollektivistische Kultur der chinesischen Kunden motiviert sie, mehrere aber kürzere \ac{OCRs} zu schreiben und beeinflusst groß die quantitativen \ac{OCRs}, damit die chinesischen Kunden mehr ``5 Sterne'' Bewertungen geben, obwohl sie nicht zufrieden sind. Deshalb sind die chinesischen quantitativen \ac{OCRs} viel positiver als die deutschen und gleichzeitig ist der Zusammenhang zwischen chinesischen quantitativen und qualitativen \ac{OCRs} schwächer. 

Mit dieser Kenntnisse sollten die Kunden beachten, dass die chinesischen Sterne (also die quantitativen \ac{OCRs}) kein glaubwürdiges Kennzeichen in der Kaufentscheidung sind, während die deutschen Sterne ein Verweis sind. Die Kunden in den beiden Ländern sollten auch mehr die schriftlichen Teile der \ac{OCRs} durchlesen, um die Glaubwürdigkeit zu erhöhen und mehrere Informationen über die Attribute des Produkts herauszufinden, damit sie die richtige Kaufentscheidung treffen können.

Die Hersteller eines Produkts sollten auch die schriftlichen Teile analysieren, um die Stärken und Schwächen durch die Rezensionen der Kunden kennenzulernen, und die auf die Probleme schnell zu reagieren. Durch diese Analyse könnten sie die durch den Kunden fokussierten Attribute erkennen und verbessern, und damit bessere Produkte oder Serviceleistungen anbieten.

Die Betreiber der Plattformen sollten diese Ergebnisse beachten. Sie sollten die \ac{OCRs} möglicherweise gut in einer bestimmten Reihenfolge (aber nicht zeitlich) ordnen, damit die nutzlosen, beispielsweise die zu kurzen \ac{OCRs}, nicht direkt an erster Stelle positioniert werden sondern die hilfreichen Kundenrezensionen, damit die anderen Kunden schnell die nützlichen Informationen finden könnten. Noch besser wäre es, wenn die Plattformen die vom Kunden fokussierten Aspekte oder die Attitüden aus den \ac{OCRs} automatisch zusammenfassten.
%%==========================================
\section{Limitationen}
%%==========================================
Die in dieser Forschung ausgegebenen Ergebnisse sollten in Verbindung mit verschiedenen Limitationen berücksichtigt werden. Diese Limitationen betrachten aus einer Vielzahl von Blickwinkeln:
\begin{enumerate}
	\item \textbf{Wirtschaftliche Limitationen}: 
	\begin{enumerate}
        	\item Diese Studie ist aus dem Bereich des Textileinzelhandels. Das bedeutet, dass sich die Ergebnisse dieser Arbeit nur auf den Textileinzelhandel beziehen. Es ist noch nachzuweisen, ob die Ergebnisse domainspezifisch sind oder nicht. Diese Untersuchungsobjekte sind die Sportkleidungen, die nicht alle Güter im Textileinzelhandel repräsentieren können.
        	\item Die in dieser Arbeit genutzten Maßnahmen sind statistisch. Diese Maßnahmen müssen zahlreiche \ac{OCRs} von gleichen Produkten in den beiden Ländern analysieren. Nur nach Kauf des Produkts können die Kunden \ac{OCRs} geben, deshalb sind die Untersuchungsobjekte beschränkt, da sie in China und Deutschland sehr gut verkauft werden müssen. Deshalb sind die \ac{OCRs} meistens positiv. 
        \end{enumerate} 
	\item \textbf{Limitationen von der Sentiment Analyse}:
	\begin{enumerate}
		\item Die in dieser Arbeit verwendete Sentiment Analyse ist nur im Dokument-Level. Es gibt auch den Aspekt-Level in Sentiment Analysen, in welchem die Analyse über die konkreten Aspekte der Kunden, zum Beispiel: wie gut ist die Kleidung in Ästhetik, Performance oder äußeren Attributen in den \ac{OCRs} berücksichtigt werden. Diese Attitüde von Kunden über die Attribute des Produkts werden in dieser Arbeit nicht studiert.
		\item Das Lexikon sollte verbessert werden. Der Algorithmus betrachtet keine Phrase, keinen Komparativ und keinen Superlativ. Die Phrase ``nicht gut'' ist natürlich kein positiver Ausdruck, und die Wörter ``besser'' und ``am besten'' sollten auch nicht gleiche Polarität wie ``gut'' haben. Aber in diesem Algorithmus werden diese Situationen ignoriert. 
	\end{enumerate}
	\item \textbf{Limitationen von Übersetzung}:
	\begin{enumerate}
		\item Der in dieser Arbeit gemachte Vergleich ist kulturübergreifend und mehrsprachig, deshalb ist es nötig eine Sprache in die andere zu übersetzen. Wegen der großen Menge der Rohdaten, wird hier eine Übersetzungsmaschine verwendet. Obwohl die Einflüsse klein sind, wie die Diskussion im Abschnitt \ref{mehrsprachig} zeigt, gibt es sie.
		\item Weil China und Deutschland unterschiedliche Kulturen haben, haben die Wörter unterschiedliche Bedeutungen. Zum Beispiel: das Wort ``Stolz'' ist in Deutschland positiv aber in China negativ, obwohl es in beiden Ländern das gleiche Wort ist.
	\end{enumerate}
	\item \textbf{Limitationen von Statistik}:
	\begin{enumerate}
		\item In dieser Arbeit, werden einige Ergebnisse im Allgemeinen herausgefunden, aber es gibt auch eine Ausnahme in den deutschen Daten aus Puma. Der Grund für diese Außerordentlichkeit ist noch nicht gefunden.  
	\end{enumerate}
\end{enumerate}
%%==========================================
\section{Zukünftige Forschung}
%%==========================================
In der Zukunft werden weiteren Forschungen mehrere Untersuchungsobjekte im Textileinzelhandel durchgeführt, um die Ergebnisse in dieser Arbeit zu beweisen. Wenn man mehrere Untersuchungsobjekte analysieren würde, könnte man auch erkennen, ob die obig genannte Ausnahme nur ein seltener Fall ist. Man sollte zwischen mehreren Ländern den kulturellen Vergleich durchführen, damit die kulturellen Einflüsse von den Dimensionen auf die \acl{OCRs} besser erkannt werden könnten. 

Für die weiteren Forschungen, ist es auch wichtig, die Limitationen von der Sentiment Analyse zu lösen. Man kann die Analyse im Aspekt-Level durchführen und herausfinden, welche Unterschiede es in den konkreten Aspekten der Kunden zwischen den Kulturen gibt, und welche Zusammenhänge es zwischen den Unterschieden und kulturellen Dimensionen von \citeauthor{hofstede2013interkulturelle} gibt. Das Lexikon und der Algorithmus sollten auch in Zukunft verbessert werden, um bessere Ergebnisse zu bekommen.

Die kulturellen Gründe werden vorgeschlagen, aber noch nicht durch Studien bewiesen. Mit den Ergebnissen dieser Arbeit kann man weitere Forschungen über die \acl{OCRs} betreiben, insbesondere eine Ursachenforschung durchführen, damit man die Zusammenhänge zwischen den kulturellen Dimensionen und den Ergebnissen bestimmen kann.

Die Motive und Wirkungen dieser Unterschiede sind auch in dieser Arbeit in der Konsumentensicht, Unternehmensicht und der Sicht der Plattformsbetreiber diskutiert geworden. Aber es ist auch wichtig die obig dargestellten Motive und Wirkungen durch weitere Forschungen zu bestimmen.

In der Zusammenfassung, macht diese Arbeit einen kulturellen Vergleich von den \acl{OCRs} im Textileinzelhandel zwischen Deutschland und China durch Sentiment Analyse, und einige Unterschiede zwischen den beiden Ländern werden in dieser Arbeit herausgefunden. Die deutschen \acl{OCRs} sind qualitativer, die chinesischen quantitativer. In dem quantitativen Aspekt sind die chinesischen \acl{OCRs} positiver aber in dem qualitativen Aspekt ist die Situation anders. Die deutschen \acl{OCRs} sind dezentralisiert und die chinesischen \acl{OCRs} sind dicht zentralisiert. Die deutschen Kunden geben die quantitativen \acl{OCRs} abhängiger von dem schriftlichen Teil, als die chinesischen. Es ist erwünscht, dass diese Ergebnisse und Diskussionen für die Konsumenten,  die Produzenten der Produkte, und die Betreiber der Plattformen im Textileinzelhandel nützlich sind.
